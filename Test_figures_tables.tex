\documentclass[12pt,]{article}
%\usepackage{lmodern}  Melissa removed to deal with font rendering issue
\usepackage{amssymb,amsmath}
\usepackage{ifxetex,ifluatex}
\usepackage{fixltx2e} % provides \textsubscript

%Melissa removed the following section to deal with font rendering issue
%\ifnum 0\ifxetex 1\fi\ifluatex 1\fi=0 % if pdftex
%  \usepackage[T1]{fontenc}
%  \usepackage[utf8]{inputenc}
%%\else % if luatex or xelatex
%  \ifxetex
%    \usepackage{mathspec}
%  \else
%    \usepackage{fontspec}
%  \fi
%  \defaultfontfeatures{Ligatures=TeX,Scale=MatchLowercase}
%  \newcommand{\euro}{€}
%%%%%%\fi

% use upquote if available, for straight quotes in verbatim environments
\IfFileExists{upquote.sty}{\usepackage{upquote}}{}
% use microtype if available
\IfFileExists{microtype.sty}{%
\usepackage{microtype}
\UseMicrotypeSet[protrusion]{basicmath} % disable protrusion for tt fonts
}{}
\usepackage[margin=1in]{geometry}
\usepackage{hyperref}
\PassOptionsToPackage{usenames,dvipsnames}{color} % color is loaded by hyperref
\hypersetup{unicode=true,
            pdfborder={0 0 0},
            breaklinks=true}
\urlstyle{same}  % don't use monospace font for urls
\setlength{\parindent}{0pt}
\setlength{\parskip}{6pt plus 2pt minus 1pt}
\setlength{\emergencystretch}{3em}  % prevent overfull lines
\providecommand{\tightlist}{%
  \setlength{\itemsep}{0pt}\setlength{\parskip}{0pt}}
\setcounter{secnumdepth}{5}

%%% Use protect on footnotes to avoid problems with footnotes in titles
\let\rmarkdownfootnote\footnote%
\def\footnote{\protect\rmarkdownfootnote}

%%% Change title format to be more compact
\usepackage{titling}

% Create subtitle command for use in maketitle
\newcommand{\subtitle}[1]{
  \posttitle{
    \begin{center}\large#1\end{center}
    }
}

\setlength{\droptitle}{-2em}
  \title{}
  \pretitle{\vspace{\droptitle}}
  \posttitle{}
  \author{}
  \preauthor{}\postauthor{}
  \date{}
  \predate{}\postdate{}


% This file contains all of the LaTeX packages you may need to compile the document
% Documentation for each package can be found onlines
\usepackage{tabularx}                                             % table environment providing flexibility
\usepackage{caption}                                              % for creating captions  
\usepackage{longtable}                                            % allows tables to span multiple pages
\usepackage{rotating}                                             % allows for sideways tables
\usepackage{float}                                                % floating environments; may not need in rmarkdown
\usepackage{placeins}                                             % keeps floats from moving
\usepackage{indentfirst}                                          % indents first paragraph of a section
\usepackage{mdwtab}                                               % continued float multi-page figure
\usepackage{enumerate}                                            % create lists
\usepackage{hyperref}                                             % highlight cross references
\hypersetup{colorlinks=true, urlcolor=blue, linktoc=page, linkcolor=blue, citecolor=blue} %define referencing colors
%\usepackage{makebox}                                             % make boxes around text
\usepackage[usenames,dvipsnames]{xcolor}                          % color name options
%\usepackage[space]{grffile}                                      % spaces in file name path
\usepackage{soul}                                                 % highlight text
\usepackage{enumitem}                                             % numbered lists
\usepackage{lineno}                                               % Line numbers; comment out for final
\usepackage{upquote}                                              % produce grave accent in latex
\usepackage{verbatim}                                             % produces verbatim results
\usepackage{fancyvrb}                                             % verbatim in a box
%\usepackage{draftwatermark}                                      % places Draft watermark in background; comment out for final
\usepackage{textcomp}                                             % fixes error with packages interfering
\usepackage{lscape}                                               % rotate pages - to allow for landscape longtables
%\pdfinterwordspaceon                                             % fix loss of inter word spacing
\usepackage{cmap}                                                 % fix mapping characters to unicode
\RequirePackage[linewidth = 1]{pdfcomment}                        % pdf comments
\RequirePackage[l2tabu, orthodox]{nag}                            % checks packages related to the accessibility?
\usepackage[inline]{showlabels}                                   % show table and figure labels; comment out for final
%\RequirePackage[tagged]{accessibilityMeta}
\usepackage{booktabs} 

\linenumbers                                                      % specify use of line numbers


\definecolor{light-gray}{gray}{.85}                               % define light-gray as a color
%\usepackage[tagged]{accessibility-meta}

 
%\showlabels[\color{mred}]{label}

% Redefines (sub)paragraphs to behave more like sections
\ifx\paragraph\undefined\else
\let\oldparagraph\paragraph
\renewcommand{\paragraph}[1]{\oldparagraph{#1}\mbox{}}
\fi
\ifx\subparagraph\undefined\else
\let\oldsubparagraph\subparagraph
\renewcommand{\subparagraph}[1]{\oldsubparagraph{#1}\mbox{}}
\fi

\begin{document}

{
\setcounter{tocdepth}{4}
\tableofcontents
}
USE THIS .Rmd TO TEST R CODE CHUNKS, FIGURES AND PLOTS BEFORE INSERTING
INTO THE MAIN TEXT OR TO DEBUG

\begin{table}[ht]
\centering
\caption{Data filtering steps for the CRFS CPFV onboard observer 
                                        index of abundance for north and south of Pt. Conception.} 
\label{tab:Fleet6_&_Filter}
\begin{tabular}{lcc}
  \hline
Filter & Drifts & Positive Drifts \\ 
  \hline
Data from SQL filtered for missing data & 15 & 15 \\ 
  Remove years prior to 2001 and north of Cape Mendocino & 14 & 14 \\ 
  Depth, remove 1\% data on each tail of positive catches & 11 & 13 \\ 
  Time fished, remove 1\% data on each tail & 10 & 12 \\ 
  Observed anglers, remove 1\% data on each tail & 9 & 11 \\ 
  Limit to reefs observering gopher/byel in at least 20 drifts & 7 & 10 \\ 
  Limite to reefs sampled in at least 2/3 of all years & 8 & 9 \\ 
  Limite to drifs within 1000 m of a reef & 6 & 8 \\ 
  Put depth in 10m depth bins, remove 0-9 and 60-69 m bins & 5 & 7 \\ 
   &  &  \\ 
  Start of north filtering & 4 & 6 \\ 
  Filter to drifts within 43 m of a reef,97\% quantile & 3 & 5 \\ 
  Make sure reefs still sampled at least 2/3 of years & 2 & 4 \\ 
   &  &  \\ 
  Start of south filtering & 1 & 3 \\ 
  Filter to drifts with $>$=20\% groundfish and recheck reefs & 13 & 2 \\ 
  Make sure reefs still sampled at least 2/3 of years & 12 & 1 \\ 
   \hline
\end{tabular}
\end{table}\begin{table}[ht]
\centering
\caption{Model selection for the CRFS CPFV onboard observer 
                                        index of abundance for north of Pt. Conception. Bold 
                                        values indicate the model selected.} 
\label{tab:Fleet6_AIC}
\begin{tabular}{lll}
  \hline
Model & Lognormal & Binomial \\ 
  \hline
Year & 14135 & 17531 \\ 
  Year + Month & 14120 & 17529 \\ 
  Year + Depth & 13953 & 17025 \\ 
  Year + Reef & 14126 & 17293 \\ 
  Year + Month + Depth & 13951 & 17027 \\ 
  Year + Month + Depth + Reef & \textbf{13921} & \textbf{16674} \\ 
   \hline
\end{tabular}
\end{table}\begin{table}[ht]
\centering
\caption{Model selection for the CRFS CPFV onboard observer 
                                        index of abundance for south of Pt. Conception. Bold 
                                        values indicate the model selected.} 
\label{tab:Fleet7_AIC}
\begin{tabular}{lll}
  \hline
Model & Lognormal & Binomial \\ 
  \hline
Year & 2798 & 5490 \\ 
  Year + Month & 2799 & 5487 \\ 
  Year + Depth & 2744 & 5159 \\ 
  Year + Reef & 2653 & 5390 \\ 
  Year + Depth + Reef & \textbf{2652} & \textbf{5071} \\ 
  Year + Depth + Reef + Month & 2663 & 5072 \\ 
   \hline
\end{tabular}
\end{table}

\end{document}
